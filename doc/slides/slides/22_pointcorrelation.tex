\begin{frame}
	\frametitle{Two Point Correlation}
	\vfill
	\begin{description}
		\item[Type:] N-Body problem;
		\item[Domain:] Points in space;
		\item[Goal:] Count the pairs of points which lie within a given radius of each other;
	\end{description}
	\vfill
	\begin{block}{Acceleration Structure}
		\begin{itemize}
			\item kD-Tree;
			\item Exclude all the points in a cell if it is too far away;
		\end{itemize}
	\end{block}
	\vfill
\end{frame}

\begin{frame}
	\frametitle{Strategy}
	\begin{itemize}\itemsep=20pt
		\item Change in the tree traversal:
		\begin{itemize}
			\item[-] \textbf{Original approach}: The tree is analysed from the root for each point;
			\item[-] \textbf{New approach}: The tree is analysed for a block of points at a time;
		\end{itemize}
		\item Spatial Sort the points by coordinates;
		\item Reduction to compute the total sum of pairs found;
	\end{itemize}
\end{frame}

\begin{frame}
	\frametitle{Expectations}
	\begin{itemize}\itemsep=20pt
		\item Fewer operations per point per node $\Rightarrow$ higher bandwidth requirements;
		\begin{itemize}
			\item The tiling transformation should reduce bus pressure;
		\end{itemize}
		\item Low amount of data per point $\Rightarrow$ larger blocks;
		\item Original work attained maximum improvements around 200\%;
	\end{itemize}
\end{frame}
