
\frame{
	\frametitle{Path Tracer}
	\vfill
	\begin{description}
		\item[Type:] Naive Ray Tracing algorithm;
		\item[Domain:] Rays colliding with spheres in a 3D space;
		\item[Goal:] Compute contribution of each path of rays, and final result (2D image);
	\end{description}
	\vfill
	\begin{block}{Acceleration Structure}
		\begin{itemize}
			\item Bounding Volume Hierarchy (for scene objects only);
			\item Optimize collision detection between rays and objects;
		\end{itemize}
	\end{block}
	\vfill
}

\frame{
	\frametitle{Strategy}
	\begin{itemize}\itemsep=20pt
		\item Change in the iteration method:
		\begin{itemize}
			\item[-] \textbf{Original approach}: Pixels are assigned to threads;
			\item[-] \textbf{New approach}: One pixel at a time. Ray blocks are assigned to threads. This increases geometrical proximity accross all current rays;
		\end{itemize}
		\item Spatial Sort all rays by origin;
		\item Spatial Sort each block by direction;
		\item Reduction to compute final pixel value;
	\end{itemize}
}

\frame{
	\frametitle{Expectations}
	\begin{itemize}\itemsep=20pt
		\item Blocking is only useful for larger scene sizes
		\begin{itemize}
			\item But that decreases coherence between rays
		\end{itemize}

		\item Divergence is much higher than N-Body problems (more difficult to keep coherent blocks)
		\begin{itemize}
			\item Solved by other algorithms (e.g. Coherent Path Tracing)
		\end{itemize}

		\item Blocks require too much temporary data $\Rightarrow$ Increased ray size, and blocks possibly too small to achieve improvements;

		\item Original work attained maximum improvements around \%;
	\end{itemize}
}