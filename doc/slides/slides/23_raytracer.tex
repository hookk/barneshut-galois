\frame{
	\frametitle{Ray Tracer}
	
	\begin{block}{}
		\begin{itemize}
			\item Simple Monte Carlo Path Tracer

			\item Based on \texttt{smallpt}: \textit{Global Illumination in 99 lines of C++}
		\end{itemize}
	\end{block}

	\begin{block}{Workflow}
		\begin{itemize}\itemsep=15pt
			\item Each pixel generates $N$ rays
			\begin{itemize}
				\item[-] $N > 10^4$
			\end{itemize}

			\item Each ray generates new sub-rays

			\item Bounding Volume Hierarchy for the scene
		\end{itemize}
	\end{block}
}

%
% Strategy
%
\frame{
	\frametitle{Ray Tracer}

	\begin{block}{Strategy}
		\begin{enumerate}\itemsep=15pt

			\item Change iteration method
			\begin{itemize}
				\item[-] \textbf{Original approach}: Distribute pixels accross threads
				\item[-] \textbf{New approach}: One pixel at a time. Distribute rays
				% NOTE: there are enough rays for this
				% NOTE: original was using OpenMP
			\end{itemize}
			
			\item Spatial Sort all rays by ray origin

			\item Spatial Sort on each block by ray direction

			\item Each ray updates itself, and keeps track of current contribution

			\item Reduction to compute final pixel color
		\end{enumerate}
	\end{block}
}

%
% Expectations
%
\frame{
	\frametitle{Ray Tracer}

	\begin{block}{Expectations}
		\begin{itemize}\itemsep=15pt

			\item Ray Tracer required more data per item, and more auxiliary data during traversal

			\item Ray Blocking can only be useful for more complex scenes. But that also increases ray divergence

			\item Spatial Sort is not enough to keep coherence between rays.
			\begin{itemize}
				\item Other strategies solve this more effectively (e.g. Coherent Path Tracing)
			\end{itemize}

			\item Original work presents Ray Tracer as the worst test case (max speedups around 10\%)
			
		\end{itemize}
	\end{block}
}