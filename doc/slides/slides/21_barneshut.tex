\begin{frame}
	\frametitle{Barnes-Hut}
	\vfill
	\begin{description}
		\item[Type:] N-Body problem;
		\item[Domain:] Bodies in space;
		\item[Goal:] Compute the evolution of a system of particles in time;
	\end{description}
	\vfill
	\begin{block}{Acceleration Structure}
		\begin{itemize}
			\item Octree;
			\item If a given cell is too far away, the cell itself is used to calculate interactions;
		\end{itemize}
	\end{block}
	\vfill
\end{frame}

\begin{frame}
	\frametitle{Strategy}
	\begin{itemize}\itemsep=20pt
		\item Change in the tree traversal:
		\begin{itemize}
			\item[-] \textbf{Original approach}: The tree is analysed from the root for each body;
			\item[-] \textbf{New approach}: The tree is analysed for a block of points at a time;
		\end{itemize}
		\item Spatial Sort the bodies by coordinates;
		\item Optimization only applied to the computation of the forces.
	\end{itemize}
\end{frame}

\begin{frame}
	\frametitle{Expectations}
	\begin{itemize}\itemsep=20pt
		\item For deep structures, sorting is unlikely to improve results;
		\item Point blocking is expected to lower considerably cache misses;
		\item Original work achieved improvements around 70\%;
	\end{itemize}
\end{frame}
